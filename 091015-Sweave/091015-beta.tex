\documentclass{article}

\usepackage{Sweave}
\begin{document}
\input{091015-beta-concordance}

\title{Dispersal Beta: Finding Ideal Parameters}
\author{Samantha L. Davis}

\maketitle

\section{Summary}
This paper explores the range of values and accuracy of the \textit{beta} (dispersal) parameter in SORTIE-ND for adult trees in our validation plots. For each set of parameters in the 081815c runs, I varied them by 10\% to test whether adjusting the parameters would increase the overall model fit. This will also give us an idea of how much swing these parameters have within the simulations. 

For each species/step combination, I'll need to evaluate whether the parameters improve or hurt the model fit. I'll be using a general linear model that regresses the expected values (the ``realPlots'' means) against the simulated values of the model. The model improves as the slope approaches 1. If realPlots data are on the y-axis, then points or lines that fall above the ``1'' demarkation line are \textit{underpredicting} the true value; and points or lines that fall below the ``1'' demarkation line are \textit{overpredicting} the true value.

We'll need to view all of the data -- data for the 90, 100, and 110 percent values of the parameters -- before we can conduct the analysis. 

View the Rnw document to view the code; otherwise, I am only printing outputs to save some space and make this document more readable.







\newpage

\section{Basal Area: At the nintieth percentile}
\begin{Schunk}
\begin{Soutput}
Call:
lm(formula = SimAbsBA ~ ExpAbsBA, data = PlotMeans)

Residuals:
     Min       1Q   Median       3Q      Max 
-25.5307   0.4451   1.8166   1.9417  11.1927 

Coefficients:
            Estimate Std. Error t value Pr(>|t|)    
(Intercept) -1.88002    0.75967  -2.475    0.015 *  
ExpAbsBA     0.98567    0.02868  34.373   <2e-16 ***
---
Signif. codes:  0 ‘***’ 0.001 ‘**’ 0.01 ‘*’ 0.05 ‘.’ 0.1 ‘ ’ 1

Residual standard error: 6.561 on 100 degrees of freedom
Multiple R-squared:  0.922,	Adjusted R-squared:  0.9212 
F-statistic:  1181 on 1 and 100 DF,  p-value: < 2.2e-16
\end{Soutput}
\end{Schunk}
\includegraphics{091015-beta-005}

Now, how are the individual species doing?

\includegraphics{091015-beta-006}
\begin{Schunk}
\begin{Sinput}
>   sppSlopes
\end{Sinput}
\begin{Soutput}
   species       ba90
1     ABCO  3.0700509
2     ABMA  0.9345975
3     CADE  1.0286246
4     PICO  0.9962182
5     PIJE -0.9722587
6     PILA  2.3945161
7     PIMO  0.9452041
8     PIPO  0.4690694
9     QUCH -0.7826302
10    QUKE  1.0128360
\end{Soutput}
\end{Schunk}


\newpage
\section{At the original parameter designation}
\begin{Schunk}
\begin{Soutput}
Call:
lm(formula = SimAbsBA ~ ExpAbsBA, data = PlotMeans)

Residuals:
     Min       1Q   Median       3Q      Max 
-24.5872   0.6374   1.8231   2.1861  11.1059 

Coefficients:
            Estimate Std. Error t value Pr(>|t|)    
(Intercept)  -1.8487     0.7418  -2.492   0.0143 *  
ExpAbsBA      1.0316     0.0293  35.208   <2e-16 ***
---
Signif. codes:  0 ‘***’ 0.001 ‘**’ 0.01 ‘*’ 0.05 ‘.’ 0.1 ‘ ’ 1

Residual standard error: 6.421 on 100 degrees of freedom
Multiple R-squared:  0.9254,	Adjusted R-squared:  0.9246 
F-statistic:  1240 on 1 and 100 DF,  p-value: < 2.2e-16
\end{Soutput}
\end{Schunk}
\includegraphics{091015-beta-008}

\includegraphics{091015-beta-009}
\begin{Schunk}
\begin{Sinput}
>   sppSlopes
\end{Sinput}
\begin{Soutput}
   species       ba90     ba100
1     ABCO  3.0700509 2.8679859
2     ABMA  0.9345975 0.8968508
3     CADE  1.0286246 0.9240230
4     PICO  0.9962182 0.9295847
5     PIJE -0.9722587 2.4811348
6     PILA  2.3945161 2.5907967
7     PIMO  0.9452041 0.9567542
8     PIPO  0.4690694 0.6919119
9     QUCH -0.7826302 7.0914178
10    QUKE  1.0128360 1.0172449
\end{Soutput}
\end{Schunk}





\newpage
\section{At the one hundred and tenth percentile}
\begin{Schunk}
\begin{Soutput}
Call:
lm(formula = SimAbsBA ~ ExpAbsBA, data = PlotMeans)

Residuals:
     Min       1Q   Median       3Q      Max 
-26.2569   0.5016   1.8413   1.9484  12.6616 

Coefficients:
            Estimate Std. Error t value Pr(>|t|)    
(Intercept) -1.88329    0.77332  -2.435   0.0166 *  
ExpAbsBA     0.98940    0.02919  33.894   <2e-16 ***
---
Signif. codes:  0 ‘***’ 0.001 ‘**’ 0.01 ‘*’ 0.05 ‘.’ 0.1 ‘ ’ 1

Residual standard error: 6.679 on 100 degrees of freedom
Multiple R-squared:  0.9199,	Adjusted R-squared:  0.9191 
F-statistic:  1149 on 1 and 100 DF,  p-value: < 2.2e-16
\end{Soutput}
\end{Schunk}
\includegraphics{091015-beta-011}

\includegraphics{091015-beta-012}
\begin{Schunk}
\begin{Sinput}
>   sppSlopes
\end{Sinput}
\begin{Soutput}
   species       ba90     ba100     ba110
1     ABCO  3.0700509 2.8679859 3.2140277
2     ABMA  0.9345975 0.8968508 0.9291240
3     CADE  1.0286246 0.9240230 1.0000032
4     PICO  0.9962182 0.9295847 1.0287976
5     PIJE -0.9722587 2.4811348 1.7046706
6     PILA  2.3945161 2.5907967 2.5827189
7     PIMO  0.9452041 0.9567542 0.9671009
8     PIPO  0.4690694 0.6919119 0.8387516
9     QUCH -0.7826302 7.0914178 1.7139325
10    QUKE  1.0128360 1.0172449 1.0054535
\end{Soutput}
\end{Schunk}






\newpage

\section{Adult Density: At the ninetieth percentile}
\begin{Schunk}
\begin{Soutput}
Call:
lm(formula = SimAbsDen ~ ExpAbsDen, data = PlotMeans)

Residuals:
     Min       1Q   Median       3Q      Max 
-1060.66  -108.88   -82.30   -20.85  1172.87 

Coefficients:
            Estimate Std. Error t value Pr(>|t|)    
(Intercept) 83.95551   32.85628   2.555   0.0121 *  
ExpAbsDen    0.39722    0.03787  10.489   <2e-16 ***
---
Signif. codes:  0 ‘***’ 0.001 ‘**’ 0.01 ‘*’ 0.05 ‘.’ 0.1 ‘ ’ 1

Residual standard error: 286.4 on 100 degrees of freedom
Multiple R-squared:  0.5238,	Adjusted R-squared:  0.5191 
F-statistic:   110 on 1 and 100 DF,  p-value: < 2.2e-16
\end{Soutput}
\end{Schunk}
\includegraphics{091015-beta-014}

Now, how are the individual species doing?

\includegraphics{091015-beta-015}
\begin{Schunk}
\begin{Sinput}
>   sppSlopes
\end{Sinput}
\begin{Soutput}
   species       ba90     ba100     ba110       den90
1     ABCO  3.0700509 2.8679859 3.2140277   0.9199434
2     ABMA  0.9345975 0.8968508 0.9291240   2.0413250
3     CADE  1.0286246 0.9240230 1.0000032   2.4190038
4     PICO  0.9962182 0.9295847 1.0287976   1.2006573
5     PIJE -0.9722587 2.4811348 1.7046706  -2.9059395
6     PILA  2.3945161 2.5907967 2.5827189   2.7015985
7     PIMO  0.9452041 0.9567542 0.9671009   3.6933377
8     PIPO  0.4690694 0.6919119 0.8387516   1.9794173
9     QUCH -0.7826302 7.0914178 1.7139325 -13.7199921
10    QUKE  1.0128360 1.0172449 1.0054535   2.6207394
\end{Soutput}
\end{Schunk}


\newpage
\section{At the original parameter designation}
\begin{Schunk}
\begin{Soutput}
Call:
lm(formula = SimAbsDen ~ ExpAbsDen, data = PlotMeans)

Residuals:
    Min      1Q  Median      3Q     Max 
-776.11  -73.57  -48.99   -4.74  687.39 

Coefficients:
            Estimate Std. Error t value Pr(>|t|)    
(Intercept) 52.13734   22.09152    2.36   0.0202 *  
ExpAbsDen    0.34780    0.02546   13.66   <2e-16 ***
---
Signif. codes:  0 ‘***’ 0.001 ‘**’ 0.01 ‘*’ 0.05 ‘.’ 0.1 ‘ ’ 1

Residual standard error: 192.5 on 100 degrees of freedom
Multiple R-squared:  0.651,	Adjusted R-squared:  0.6475 
F-statistic: 186.6 on 1 and 100 DF,  p-value: < 2.2e-16
\end{Soutput}
\end{Schunk}
\includegraphics{091015-beta-017}

\includegraphics{091015-beta-018}
\begin{Schunk}
\begin{Sinput}
>   sppSlopes
\end{Sinput}
\begin{Soutput}
   species       ba90     ba100     ba110       den90     den100
1     ABCO  3.0700509 2.8679859 3.2140277   0.9199434   1.366993
2     ABMA  0.9345975 0.8968508 0.9291240   2.0413250   2.218563
3     CADE  1.0286246 0.9240230 1.0000032   2.4190038   2.345902
4     PICO  0.9962182 0.9295847 1.0287976   1.2006573   1.479854
5     PIJE -0.9722587 2.4811348 1.7046706  -2.9059395  27.477350
6     PILA  2.3945161 2.5907967 2.5827189   2.7015985   3.557684
7     PIMO  0.9452041 0.9567542 0.9671009   3.6933377   4.056143
8     PIPO  0.4690694 0.6919119 0.8387516   1.9794173   1.296860
9     QUCH -0.7826302 7.0914178 1.7139325 -13.7199921 -15.618990
10    QUKE  1.0128360 1.0172449 1.0054535   2.6207394   2.577005
\end{Soutput}
\end{Schunk}





\newpage
\section{At the one hundred and tenth percentile}
\begin{Schunk}
\begin{Soutput}
Call:
lm(formula = SimAbsDen ~ ExpAbsDen, data = PlotMeans)

Residuals:
    Min      1Q  Median      3Q     Max 
-2517.0  -363.4  -298.4  -217.8  2984.8 

Coefficients:
            Estimate Std. Error t value Pr(>|t|)    
(Intercept) 289.5353   101.0477   2.865  0.00508 ** 
ExpAbsDen     0.6811     0.1165   5.848 6.29e-08 ***
---
Signif. codes:  0 ‘***’ 0.001 ‘**’ 0.01 ‘*’ 0.05 ‘.’ 0.1 ‘ ’ 1

Residual standard error: 880.7 on 100 degrees of freedom
Multiple R-squared:  0.2548,	Adjusted R-squared:  0.2474 
F-statistic:  34.2 on 1 and 100 DF,  p-value: 6.289e-08
\end{Soutput}
\end{Schunk}
\includegraphics{091015-beta-020}

\includegraphics{091015-beta-021}
\begin{Schunk}
\begin{Sinput}
>   sppSlopes
\end{Sinput}
\begin{Soutput}
   species       ba90     ba100     ba110       den90     den100      den110
1     ABCO  3.0700509 2.8679859 3.2140277   0.9199434   1.366993   0.2191432
2     ABMA  0.9345975 0.8968508 0.9291240   2.0413250   2.218563   0.7642356
3     CADE  1.0286246 0.9240230 1.0000032   2.4190038   2.345902   2.2804536
4     PICO  0.9962182 0.9295847 1.0287976   1.2006573   1.479854   0.5540442
5     PIJE -0.9722587 2.4811348 1.7046706  -2.9059395  27.477350  -0.6207801
6     PILA  2.3945161 2.5907967 2.5827189   2.7015985   3.557684   1.0074069
7     PIMO  0.9452041 0.9567542 0.9671009   3.6933377   4.056143   2.3065268
8     PIPO  0.4690694 0.6919119 0.8387516   1.9794173   1.296860  -0.2537711
9     QUCH -0.7826302 7.0914178 1.7139325 -13.7199921 -15.618990 -14.0318101
10    QUKE  1.0128360 1.0172449 1.0054535   2.6207394   2.577005   2.5996866
\end{Soutput}
\begin{Sinput}
>     write.csv(sppSlopes, file=paste(parName, ".csv", sep=""))
\end{Sinput}
\end{Schunk}
\end{document}
