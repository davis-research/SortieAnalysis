\documentclass{article}

\usepackage{Sweave}
\begin{document}
\input{091015-sm-concordance}

\title{Seedling and Sapling Mortality: Finding Ideal Parameters}
\author{Samantha L. Davis}

\maketitle

\section{Summary}
This paper explores the range of values and accuracy of the \textit{Seedling and Sapling Mortality} parameter in SORTIE-ND for adult trees in our validation plots. For each set of parameters in the 081815c runs, I varied them by 10\% to test whether adjusting the parameters would increase the overall model fit. This will also give us an idea of how much swing these parameters have within the simulations. 

For each species/step combination, I'll need to evaluate whether the parameters improve or hurt the model fit. I'll be using a general linear model that regresses the expected values (the ``realPlots'' means) against the simulated values of the model. The model improves as the slope approaches 1. If realPlots data are on the y-axis, then points or lines that fall above the ``1'' demarkation line are \textit{underpredicting} the true value; and points or lines that fall below the ``1'' demarkation line are \textit{overpredicting} the true value.

We'll need to view all of the data -- data for the 90, 100, and 110 percent values of the parameters -- before we can conduct the analysis. 

View the Rnw document to view the code; otherwise, I am only printing outputs to save some space and make this document more readable.







\newpage

\section{Basal Area: At the nintieth percentile}
\begin{Schunk}
\begin{Soutput}
Call:
lm(formula = SimAbsBA ~ ExpAbsBA, data = PlotMeans)

Residuals:
     Min       1Q   Median       3Q      Max 
-24.9126   0.5499   1.9937   2.1440  12.1954 

Coefficients:
            Estimate Std. Error t value Pr(>|t|)    
(Intercept) -2.10150    0.75937  -2.767  0.00673 ** 
ExpAbsBA     0.99359    0.02866  34.663  < 2e-16 ***
---
Signif. codes:  0 ‘***’ 0.001 ‘**’ 0.01 ‘*’ 0.05 ‘.’ 0.1 ‘ ’ 1

Residual standard error: 6.558 on 100 degrees of freedom
Multiple R-squared:  0.9232,	Adjusted R-squared:  0.9224 
F-statistic:  1202 on 1 and 100 DF,  p-value: < 2.2e-16
\end{Soutput}
\end{Schunk}
\includegraphics{091015-sm-005}

Now, how are the individual species doing?

\includegraphics{091015-sm-006}
\begin{Schunk}
\begin{Sinput}
>   sppSlopes
\end{Sinput}
\begin{Soutput}
   species      ba90
1     ABCO 2.8766174
2     ABMA 0.9271300
3     CADE 1.0052458
4     PICO 0.9819008
5     PIJE 2.2122491
6     PILA 2.7750609
7     PIMO 1.1395556
8     PIPO 0.5983266
9     QUCH 2.3647787
10    QUKE 1.0449249
\end{Soutput}
\end{Schunk}


\newpage
\section{At the original parameter designation}
\begin{Schunk}
\begin{Soutput}
Call:
lm(formula = SimAbsBA ~ ExpAbsBA, data = PlotMeans)

Residuals:
     Min       1Q   Median       3Q      Max 
-24.5872   0.6374   1.8231   2.1861  11.1059 

Coefficients:
            Estimate Std. Error t value Pr(>|t|)    
(Intercept)  -1.8487     0.7418  -2.492   0.0143 *  
ExpAbsBA      1.0316     0.0293  35.208   <2e-16 ***
---
Signif. codes:  0 ‘***’ 0.001 ‘**’ 0.01 ‘*’ 0.05 ‘.’ 0.1 ‘ ’ 1

Residual standard error: 6.421 on 100 degrees of freedom
Multiple R-squared:  0.9254,	Adjusted R-squared:  0.9246 
F-statistic:  1240 on 1 and 100 DF,  p-value: < 2.2e-16
\end{Soutput}
\end{Schunk}
\includegraphics{091015-sm-008}

\includegraphics{091015-sm-009}
\begin{Schunk}
\begin{Sinput}
>   sppSlopes
\end{Sinput}
\begin{Soutput}
   species      ba90     ba100
1     ABCO 2.8766174 2.8679859
2     ABMA 0.9271300 0.8968508
3     CADE 1.0052458 0.9240230
4     PICO 0.9819008 0.9295847
5     PIJE 2.2122491 2.4811348
6     PILA 2.7750609 2.5907967
7     PIMO 1.1395556 0.9567542
8     PIPO 0.5983266 0.6919119
9     QUCH 2.3647787 7.0914178
10    QUKE 1.0449249 1.0172449
\end{Soutput}
\end{Schunk}





\newpage
\section{At the one hundred and tenth percentile}
\begin{Schunk}
\begin{Soutput}
Call:
lm(formula = SimAbsBA ~ ExpAbsBA, data = PlotMeans)

Residuals:
     Min       1Q   Median       3Q      Max 
-25.9582   0.8691   1.8924   2.0398  11.3801 

Coefficients:
            Estimate Std. Error t value Pr(>|t|)    
(Intercept) -1.98195    0.77226  -2.566   0.0118 *  
ExpAbsBA     0.98767    0.02915  33.882   <2e-16 ***
---
Signif. codes:  0 ‘***’ 0.001 ‘**’ 0.01 ‘*’ 0.05 ‘.’ 0.1 ‘ ’ 1

Residual standard error: 6.67 on 100 degrees of freedom
Multiple R-squared:  0.9199,	Adjusted R-squared:  0.9191 
F-statistic:  1148 on 1 and 100 DF,  p-value: < 2.2e-16
\end{Soutput}
\end{Schunk}
\includegraphics{091015-sm-011}

\includegraphics{091015-sm-012}
\begin{Schunk}
\begin{Sinput}
>   sppSlopes
\end{Sinput}
\begin{Soutput}
   species      ba90     ba100     ba110
1     ABCO 2.8766174 2.8679859 3.1357509
2     ABMA 0.9271300 0.8968508 0.9311750
3     CADE 1.0052458 0.9240230 0.9966785
4     PICO 0.9819008 0.9295847 0.9463342
5     PIJE 2.2122491 2.4811348 2.7615084
6     PILA 2.7750609 2.5907967 2.8381504
7     PIMO 1.1395556 0.9567542 1.0481454
8     PIPO 0.5983266 0.6919119 0.6280481
9     QUCH 2.3647787 7.0914178 1.0675412
10    QUKE 1.0449249 1.0172449 1.0323242
\end{Soutput}
\end{Schunk}






\newpage

\section{Adult Density: At the ninetieth percentile}
\begin{Schunk}
\begin{Soutput}
Call:
lm(formula = SimAbsDen ~ ExpAbsDen, data = PlotMeans)

Residuals:
    Min      1Q  Median      3Q     Max 
-810.53  -77.19  -52.97   -8.45  800.04 

Coefficients:
            Estimate Std. Error t value Pr(>|t|)    
(Intercept) 55.70159   23.78273   2.342   0.0212 *  
ExpAbsDen    0.35802    0.02741  13.060   <2e-16 ***
---
Signif. codes:  0 ‘***’ 0.001 ‘**’ 0.01 ‘*’ 0.05 ‘.’ 0.1 ‘ ’ 1

Residual standard error: 207.3 on 100 degrees of freedom
Multiple R-squared:  0.6304,	Adjusted R-squared:  0.6267 
F-statistic: 170.6 on 1 and 100 DF,  p-value: < 2.2e-16
\end{Soutput}
\end{Schunk}
\includegraphics{091015-sm-014}

Now, how are the individual species doing?

\includegraphics{091015-sm-015}
\begin{Schunk}
\begin{Sinput}
>   sppSlopes
\end{Sinput}
\begin{Soutput}
   species      ba90     ba100     ba110      den90
1     ABCO 2.8766174 2.8679859 3.1357509   1.220608
2     ABMA 0.9271300 0.8968508 0.9311750   2.202152
3     CADE 1.0052458 0.9240230 0.9966785   2.296417
4     PICO 0.9819008 0.9295847 0.9463342   1.521204
5     PIJE 2.2122491 2.4811348 2.7615084 -13.629032
6     PILA 2.7750609 2.5907967 2.8381504   3.549251
7     PIMO 1.1395556 0.9567542 1.0481454   4.031545
8     PIPO 0.5983266 0.6919119 0.6280481   1.134669
9     QUCH 2.3647787 7.0914178 1.0675412 -23.467487
10    QUKE 1.0449249 1.0172449 1.0323242   2.591498
\end{Soutput}
\end{Schunk}


\newpage
\section{At the original parameter designation}
\begin{Schunk}
\begin{Soutput}
Call:
lm(formula = SimAbsDen ~ ExpAbsDen, data = PlotMeans)

Residuals:
    Min      1Q  Median      3Q     Max 
-776.11  -73.57  -48.99   -4.74  687.39 

Coefficients:
            Estimate Std. Error t value Pr(>|t|)    
(Intercept) 52.13734   22.09152    2.36   0.0202 *  
ExpAbsDen    0.34780    0.02546   13.66   <2e-16 ***
---
Signif. codes:  0 ‘***’ 0.001 ‘**’ 0.01 ‘*’ 0.05 ‘.’ 0.1 ‘ ’ 1

Residual standard error: 192.5 on 100 degrees of freedom
Multiple R-squared:  0.651,	Adjusted R-squared:  0.6475 
F-statistic: 186.6 on 1 and 100 DF,  p-value: < 2.2e-16
\end{Soutput}
\end{Schunk}
\includegraphics{091015-sm-017}

\includegraphics{091015-sm-018}
\begin{Schunk}
\begin{Sinput}
>   sppSlopes
\end{Sinput}
\begin{Soutput}
   species      ba90     ba100     ba110      den90     den100
1     ABCO 2.8766174 2.8679859 3.1357509   1.220608   1.366993
2     ABMA 0.9271300 0.8968508 0.9311750   2.202152   2.218563
3     CADE 1.0052458 0.9240230 0.9966785   2.296417   2.345902
4     PICO 0.9819008 0.9295847 0.9463342   1.521204   1.479854
5     PIJE 2.2122491 2.4811348 2.7615084 -13.629032  27.477350
6     PILA 2.7750609 2.5907967 2.8381504   3.549251   3.557684
7     PIMO 1.1395556 0.9567542 1.0481454   4.031545   4.056143
8     PIPO 0.5983266 0.6919119 0.6280481   1.134669   1.296860
9     QUCH 2.3647787 7.0914178 1.0675412 -23.467487 -15.618990
10    QUKE 1.0449249 1.0172449 1.0323242   2.591498   2.577005
\end{Soutput}
\end{Schunk}





\newpage
\section{At the one hundred and tenth percentile}
\begin{Schunk}
\begin{Soutput}
Call:
lm(formula = SimAbsDen ~ ExpAbsDen, data = PlotMeans)

Residuals:
    Min      1Q  Median      3Q     Max 
-753.40  -69.28  -46.22    2.88  626.56 

Coefficients:
            Estimate Std. Error t value Pr(>|t|)    
(Intercept) 49.11265   21.05987   2.332   0.0217 *  
ExpAbsDen    0.34163    0.02427  14.074   <2e-16 ***
---
Signif. codes:  0 ‘***’ 0.001 ‘**’ 0.01 ‘*’ 0.05 ‘.’ 0.1 ‘ ’ 1

Residual standard error: 183.6 on 100 degrees of freedom
Multiple R-squared:  0.6645,	Adjusted R-squared:  0.6611 
F-statistic: 198.1 on 1 and 100 DF,  p-value: < 2.2e-16
\end{Soutput}
\end{Schunk}
\includegraphics{091015-sm-020}

\includegraphics{091015-sm-021}
\begin{Schunk}
\begin{Sinput}
>   sppSlopes
\end{Sinput}
\begin{Soutput}
   species      ba90     ba100     ba110      den90     den100      den110
1     ABCO 2.8766174 2.8679859 3.1357509   1.220608   1.366993   1.4889018
2     ABMA 0.9271300 0.8968508 0.9311750   2.202152   2.218563   2.2226695
3     CADE 1.0052458 0.9240230 0.9966785   2.296417   2.345902   2.3562069
4     PICO 0.9819008 0.9295847 0.9463342   1.521204   1.479854   1.5199374
5     PIJE 2.2122491 2.4811348 2.7615084 -13.629032  27.477350  13.0711825
6     PILA 2.7750609 2.5907967 2.8381504   3.549251   3.557684   3.6246472
7     PIMO 1.1395556 0.9567542 1.0481454   4.031545   4.056143   4.1586607
8     PIPO 0.5983266 0.6919119 0.6280481   1.134669   1.296860   0.4535111
9     QUCH 2.3647787 7.0914178 1.0675412 -23.467487 -15.618990 -24.5395020
10    QUKE 1.0449249 1.0172449 1.0323242   2.591498   2.577005   2.6338477
\end{Soutput}
\begin{Sinput}
>     write.csv(sppSlopes, file=paste(parName, ".csv", sep=""))
\end{Sinput}
\end{Schunk}
\end{document}
