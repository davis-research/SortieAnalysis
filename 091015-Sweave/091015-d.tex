\documentclass{article}

\usepackage{Sweave}
\begin{document}
\input{091015-d-concordance}

\title{Dispersal Parameter D: Finding Ideal Parameters}
\author{Samantha L. Davis}

\maketitle

\section{Summary}
This paper explores the range of values and accuracy of the \textit{d} (dispersal) parameter in SORTIE-ND for adult trees in our validation plots. For each set of parameters in the 081815c runs, I varied them by 10\% to test whether adjusting the parameters would increase the overall model fit. This will also give us an idea of how much swing these parameters have within the simulations. 

For each species/step combination, I'll need to evaluate whether the parameters improve or hurt the model fit. I'll be using a general linear model that regresses the expected values (the ``realPlots'' means) against the simulated values of the model. The model improves as the slope approaches 1. If realPlots data are on the y-axis, then points or lines that fall above the ``1'' demarkation line are \textit{underpredicting} the true value; and points or lines that fall below the ``1'' demarkation line are \textit{overpredicting} the true value.

We'll need to view all of the data -- data for the 90, 100, and 110 percent values of the parameters -- before we can conduct the analysis. 

View the Rnw document to view the code; otherwise, I am only printing outputs to save some space and make this document more readable.







\newpage

\section{Basal Area: At the nintieth percentile}
\begin{Schunk}
\begin{Soutput}
Call:
lm(formula = SimAbsBA ~ ExpAbsBA, data = PlotMeans)

Residuals:
     Min       1Q   Median       3Q      Max 
-26.1942  -0.0346   2.0116   2.2577  13.1676 

Coefficients:
            Estimate Std. Error t value Pr(>|t|)    
(Intercept) -2.20785    0.83476  -2.645  0.00955 ** 
ExpAbsBA     1.00789    0.03131  32.190  < 2e-16 ***
---
Signif. codes:  0 ‘***’ 0.001 ‘**’ 0.01 ‘*’ 0.05 ‘.’ 0.1 ‘ ’ 1

Residual standard error: 6.982 on 96 degrees of freedom
Multiple R-squared:  0.9152,	Adjusted R-squared:  0.9143 
F-statistic:  1036 on 1 and 96 DF,  p-value: < 2.2e-16
\end{Soutput}
\end{Schunk}
\includegraphics{091015-d-005}

Now, how are the individual species doing?

\includegraphics{091015-d-006}
\begin{Schunk}
\begin{Sinput}
>   sppSlopes
\end{Sinput}
\begin{Soutput}
   species       ba90
1     ABCO  3.2109158
2     ABMA  0.9094796
3     CADE  0.9970746
4     PICO  0.9559808
5     PIJE  2.3726643
6     PILA  2.7357797
7     PIMO  2.8469262
8     PIPO  0.5107506
9     QUCH 21.0878545
10    QUKE  1.0346412
\end{Soutput}
\end{Schunk}


\newpage
\section{At the original parameter designation}
\begin{Schunk}
\begin{Soutput}
Call:
lm(formula = SimAbsBA ~ ExpAbsBA, data = PlotMeans)

Residuals:
     Min       1Q   Median       3Q      Max 
-24.5872   0.6374   1.8231   2.1861  11.1059 

Coefficients:
            Estimate Std. Error t value Pr(>|t|)    
(Intercept)  -1.8487     0.7418  -2.492   0.0143 *  
ExpAbsBA      1.0316     0.0293  35.208   <2e-16 ***
---
Signif. codes:  0 ‘***’ 0.001 ‘**’ 0.01 ‘*’ 0.05 ‘.’ 0.1 ‘ ’ 1

Residual standard error: 6.421 on 100 degrees of freedom
Multiple R-squared:  0.9254,	Adjusted R-squared:  0.9246 
F-statistic:  1240 on 1 and 100 DF,  p-value: < 2.2e-16
\end{Soutput}
\end{Schunk}
\includegraphics{091015-d-008}

\includegraphics{091015-d-009}
\begin{Schunk}
\begin{Sinput}
>   sppSlopes
\end{Sinput}
\begin{Soutput}
   species       ba90     ba100
1     ABCO  3.2109158 2.8679859
2     ABMA  0.9094796 0.8968508
3     CADE  0.9970746 0.9240230
4     PICO  0.9559808 0.9295847
5     PIJE  2.3726643 2.4811348
6     PILA  2.7357797 2.5907967
7     PIMO  2.8469262 0.9567542
8     PIPO  0.5107506 0.6919119
9     QUCH 21.0878545 7.0914178
10    QUKE  1.0346412 1.0172449
\end{Soutput}
\end{Schunk}





\newpage
\section{At the one hundred and tenth percentile}
\begin{Schunk}
\begin{Soutput}
Call:
lm(formula = SimAbsBA ~ ExpAbsBA, data = PlotMeans)

Residuals:
     Min       1Q   Median       3Q      Max 
-25.8152   0.8784   1.7711   1.8960  11.4314 

Coefficients:
            Estimate Std. Error t value Pr(>|t|)    
(Intercept) -1.84186    0.77383   -2.38   0.0192 *  
ExpAbsBA     0.98839    0.02921   33.84   <2e-16 ***
---
Signif. codes:  0 ‘***’ 0.001 ‘**’ 0.01 ‘*’ 0.05 ‘.’ 0.1 ‘ ’ 1

Residual standard error: 6.683 on 100 degrees of freedom
Multiple R-squared:  0.9197,	Adjusted R-squared:  0.9189 
F-statistic:  1145 on 1 and 100 DF,  p-value: < 2.2e-16
\end{Soutput}
\end{Schunk}
\includegraphics{091015-d-011}

\includegraphics{091015-d-012}
\begin{Schunk}
\begin{Sinput}
>   sppSlopes
\end{Sinput}
\begin{Soutput}
   species       ba90     ba100     ba110
1     ABCO  3.2109158 2.8679859 3.1087661
2     ABMA  0.9094796 0.8968508 0.9307194
3     CADE  0.9970746 0.9240230 0.9920768
4     PICO  0.9559808 0.9295847 0.9704309
5     PIJE  2.3726643 2.4811348 2.2106834
6     PILA  2.7357797 2.5907967 2.7875110
7     PIMO  2.8469262 0.9567542 0.9114492
8     PIPO  0.5107506 0.6919119 0.5507543
9     QUCH 21.0878545 7.0914178 5.3437429
10    QUKE  1.0346412 1.0172449 1.0437060
\end{Soutput}
\end{Schunk}






\newpage

\section{Adult Density: At the ninetieth percentile}
\begin{Schunk}
\begin{Soutput}
Call:
lm(formula = SimAbsDen ~ ExpAbsDen, data = PlotMeans)

Residuals:
    Min      1Q  Median      3Q     Max 
-762.49  -73.64  -49.40   -6.59  685.36 

Coefficients:
            Estimate Std. Error t value Pr(>|t|)    
(Intercept) 52.40326   22.34659   2.345    0.021 *  
ExpAbsDen    0.34899    0.02576  13.549   <2e-16 ***
---
Signif. codes:  0 ‘***’ 0.001 ‘**’ 0.01 ‘*’ 0.05 ‘.’ 0.1 ‘ ’ 1

Residual standard error: 194.8 on 100 degrees of freedom
Multiple R-squared:  0.6474,	Adjusted R-squared:  0.6438 
F-statistic: 183.6 on 1 and 100 DF,  p-value: < 2.2e-16
\end{Soutput}
\end{Schunk}
\includegraphics{091015-d-014}

Now, how are the individual species doing?

\includegraphics{091015-d-015}
\begin{Schunk}
\begin{Sinput}
>   sppSlopes
\end{Sinput}
\begin{Soutput}
   species       ba90     ba100     ba110       den90
1     ABCO  3.2109158 2.8679859 3.1087661   1.3363986
2     ABMA  0.9094796 0.8968508 0.9307194   2.2171767
3     CADE  0.9970746 0.9240230 0.9920768   2.3013607
4     PICO  0.9559808 0.9295847 0.9704309   1.5582780
5     PIJE  2.3726643 2.4811348 2.2106834  15.4895756
6     PILA  2.7357797 2.5907967 2.7875110   3.6350821
7     PIMO  2.8469262 0.9567542 0.9114492   4.0129953
8     PIPO  0.5107506 0.6919119 0.5507543   0.9001013
9     QUCH 21.0878545 7.0914178 5.3437429 -19.6701429
10    QUKE  1.0346412 1.0172449 1.0437060   2.5772572
\end{Soutput}
\end{Schunk}


\newpage
\section{At the original parameter designation}
\begin{Schunk}
\begin{Soutput}
Call:
lm(formula = SimAbsDen ~ ExpAbsDen, data = PlotMeans)

Residuals:
    Min      1Q  Median      3Q     Max 
-776.11  -73.57  -48.99   -4.74  687.39 

Coefficients:
            Estimate Std. Error t value Pr(>|t|)    
(Intercept) 52.13734   22.09152    2.36   0.0202 *  
ExpAbsDen    0.34780    0.02546   13.66   <2e-16 ***
---
Signif. codes:  0 ‘***’ 0.001 ‘**’ 0.01 ‘*’ 0.05 ‘.’ 0.1 ‘ ’ 1

Residual standard error: 192.5 on 100 degrees of freedom
Multiple R-squared:  0.651,	Adjusted R-squared:  0.6475 
F-statistic: 186.6 on 1 and 100 DF,  p-value: < 2.2e-16
\end{Soutput}
\end{Schunk}
\includegraphics{091015-d-017}

\includegraphics{091015-d-018}
\begin{Schunk}
\begin{Sinput}
>   sppSlopes
\end{Sinput}
\begin{Soutput}
   species       ba90     ba100     ba110       den90     den100
1     ABCO  3.2109158 2.8679859 3.1087661   1.3363986   1.366993
2     ABMA  0.9094796 0.8968508 0.9307194   2.2171767   2.218563
3     CADE  0.9970746 0.9240230 0.9920768   2.3013607   2.345902
4     PICO  0.9559808 0.9295847 0.9704309   1.5582780   1.479854
5     PIJE  2.3726643 2.4811348 2.2106834  15.4895756  27.477350
6     PILA  2.7357797 2.5907967 2.7875110   3.6350821   3.557684
7     PIMO  2.8469262 0.9567542 0.9114492   4.0129953   4.056143
8     PIPO  0.5107506 0.6919119 0.5507543   0.9001013   1.296860
9     QUCH 21.0878545 7.0914178 5.3437429 -19.6701429 -15.618990
10    QUKE  1.0346412 1.0172449 1.0437060   2.5772572   2.577005
\end{Soutput}
\end{Schunk}





\newpage
\section{At the one hundred and tenth percentile}
\begin{Schunk}
\begin{Soutput}
Call:
lm(formula = SimAbsDen ~ ExpAbsDen, data = PlotMeans)

Residuals:
    Min      1Q  Median      3Q     Max 
-753.63  -73.21  -49.58   -7.48  693.39 

Coefficients:
            Estimate Std. Error t value Pr(>|t|)    
(Intercept) 52.20678   22.07180   2.365   0.0199 *  
ExpAbsDen    0.35006    0.02544  13.759   <2e-16 ***
---
Signif. codes:  0 ‘***’ 0.001 ‘**’ 0.01 ‘*’ 0.05 ‘.’ 0.1 ‘ ’ 1

Residual standard error: 192.4 on 100 degrees of freedom
Multiple R-squared:  0.6544,	Adjusted R-squared:  0.6509 
F-statistic: 189.3 on 1 and 100 DF,  p-value: < 2.2e-16
\end{Soutput}
\end{Schunk}
\includegraphics{091015-d-020}

\includegraphics{091015-d-021}
\begin{Schunk}
\begin{Sinput}
>   sppSlopes
\end{Sinput}
\begin{Soutput}
   species       ba90     ba100     ba110       den90     den100      den110
1     ABCO  3.2109158 2.8679859 3.1087661   1.3363986   1.366993   1.3618762
2     ABMA  0.9094796 0.8968508 0.9307194   2.2171767   2.218563   2.2173498
3     CADE  0.9970746 0.9240230 0.9920768   2.3013607   2.345902   2.3009673
4     PICO  0.9559808 0.9295847 0.9704309   1.5582780   1.479854   1.4610254
5     PIJE  2.3726643 2.4811348 2.2106834  15.4895756  27.477350 -15.3898886
6     PILA  2.7357797 2.5907967 2.7875110   3.6350821   3.557684   3.4773984
7     PIMO  2.8469262 0.9567542 0.9114492   4.0129953   4.056143   3.9741788
8     PIPO  0.5107506 0.6919119 0.5507543   0.9001013   1.296860   0.3865049
9     QUCH 21.0878545 7.0914178 5.3437429 -19.6701429 -15.618990 -14.8066538
10    QUKE  1.0346412 1.0172449 1.0437060   2.5772572   2.577005   2.5995907
\end{Soutput}
\begin{Sinput}
>     write.csv(sppSlopes, file=paste(parName, ".csv", sep=""))
\end{Sinput}
\end{Schunk}
\end{document}
