\documentclass{article}

\usepackage{Sweave}
\begin{document}
\input{091015-g-concordance}

\title{Growth: Finding Ideal Parameters}
\author{Samantha L. Davis}

\maketitle

\section{Summary}
This paper explores the range of values and accuracy of the \textit{mean growth} parameter in SORTIE-ND for adult trees in our validation plots. For each set of parameters in the 081815c runs, I varied them by 10\% to test whether adjusting the parameters would increase the overall model fit. This will also give us an idea of how much swing these parameters have within the simulations. 

For each species/step combination, I'll need to evaluate whether the parameters improve or hurt the model fit. I'll be using a general linear model that regresses the expected values (the ``realPlots'' means) against the simulated values of the model. The model improves as the slope approaches 1. If realPlots data are on the y-axis, then points or lines that fall above the ``1'' demarkation line are \textit{underpredicting} the true value; and points or lines that fall below the ``1'' demarkation line are \textit{overpredicting} the true value.

We'll need to view all of the data -- data for the 90, 100, and 110 percent values of the parameters -- before we can conduct the analysis. 

View the Rnw document to view the code; otherwise, I am only printing outputs to save some space and make this document more readable.







\newpage

\section{Basal Area: At the nintieth percentile}
\begin{Schunk}
\begin{Soutput}
Call:
lm(formula = SimAbsBA ~ ExpAbsBA, data = PlotMeans)

Residuals:
     Min       1Q   Median       3Q      Max 
-23.4442   0.3055   1.8349   2.2888  15.8144 

Coefficients:
            Estimate Std. Error t value Pr(>|t|)    
(Intercept) -1.92060    0.80834  -2.376   0.0196 *  
ExpAbsBA     0.94029    0.03095  30.384   <2e-16 ***
---
Signif. codes:  0 ‘***’ 0.001 ‘**’ 0.01 ‘*’ 0.05 ‘.’ 0.1 ‘ ’ 1

Residual standard error: 6.478 on 91 degrees of freedom
Multiple R-squared:  0.9103,	Adjusted R-squared:  0.9093 
F-statistic: 923.2 on 1 and 91 DF,  p-value: < 2.2e-16
\end{Soutput}
\end{Schunk}
\includegraphics{091015-g-005}

Now, how are the individual species doing?

\includegraphics{091015-g-006}
\begin{Schunk}
\begin{Sinput}
>   sppSlopes
\end{Sinput}
\begin{Soutput}
   species       ba90
1     ABCO  3.0579913
2     ABMA  0.9690096
3     CADE  1.0242647
4     PICO 12.5267866
5     PIJE  2.6504097
6     PILA  2.6446403
7     PIMO -0.1743235
8     PIPO  0.5949587
9     QUCH  0.5687714
10    QUKE  1.0520175
\end{Soutput}
\end{Schunk}


\newpage
\section{At the original parameter designation}
\begin{Schunk}
\begin{Soutput}
Call:
lm(formula = SimAbsBA ~ ExpAbsBA, data = PlotMeans)

Residuals:
     Min       1Q   Median       3Q      Max 
-24.5872   0.6374   1.8231   2.1861  11.1059 

Coefficients:
            Estimate Std. Error t value Pr(>|t|)    
(Intercept)  -1.8487     0.7418  -2.492   0.0143 *  
ExpAbsBA      1.0316     0.0293  35.208   <2e-16 ***
---
Signif. codes:  0 ‘***’ 0.001 ‘**’ 0.01 ‘*’ 0.05 ‘.’ 0.1 ‘ ’ 1

Residual standard error: 6.421 on 100 degrees of freedom
Multiple R-squared:  0.9254,	Adjusted R-squared:  0.9246 
F-statistic:  1240 on 1 and 100 DF,  p-value: < 2.2e-16
\end{Soutput}
\end{Schunk}
\includegraphics{091015-g-008}

\includegraphics{091015-g-009}
\begin{Schunk}
\begin{Sinput}
>   sppSlopes
\end{Sinput}
\begin{Soutput}
   species       ba90     ba100
1     ABCO  3.0579913 2.8679859
2     ABMA  0.9690096 0.8968508
3     CADE  1.0242647 0.9240230
4     PICO 12.5267866 0.9295847
5     PIJE  2.6504097 2.4811348
6     PILA  2.6446403 2.5907967
7     PIMO -0.1743235 0.9567542
8     PIPO  0.5949587 0.6919119
9     QUCH  0.5687714 7.0914178
10    QUKE  1.0520175 1.0172449
\end{Soutput}
\end{Schunk}





\newpage
\section{At the one hundred and tenth percentile}
\begin{Schunk}
\begin{Soutput}
Call:
lm(formula = SimAbsBA ~ ExpAbsBA, data = PlotMeans)

Residuals:
    Min      1Q  Median      3Q     Max 
-25.730   0.686   1.782   1.944  11.360 

Coefficients:
            Estimate Std. Error t value Pr(>|t|)    
(Intercept) -1.84426    0.76691  -2.405    0.018 *  
ExpAbsBA     0.99195    0.02895  34.266   <2e-16 ***
---
Signif. codes:  0 ‘***’ 0.001 ‘**’ 0.01 ‘*’ 0.05 ‘.’ 0.1 ‘ ’ 1

Residual standard error: 6.623 on 100 degrees of freedom
Multiple R-squared:  0.9215,	Adjusted R-squared:  0.9207 
F-statistic:  1174 on 1 and 100 DF,  p-value: < 2.2e-16
\end{Soutput}
\end{Schunk}
\includegraphics{091015-g-011}

\includegraphics{091015-g-012}
\begin{Schunk}
\begin{Sinput}
>   sppSlopes
\end{Sinput}
\begin{Soutput}
   species       ba90     ba100       ba110
1     ABCO  3.0579913 2.8679859   2.9938315
2     ABMA  0.9690096 0.8968508   0.9284805
3     CADE  1.0242647 0.9240230   0.9865699
4     PICO 12.5267866 0.9295847   0.9950835
5     PIJE  2.6504097 2.4811348   3.1278394
6     PILA  2.6446403 2.5907967   2.5950433
7     PIMO -0.1743235 0.9567542   0.9179473
8     PIPO  0.5949587 0.6919119   0.6106893
9     QUCH  0.5687714 7.0914178 -12.7792250
10    QUKE  1.0520175 1.0172449   1.0209050
\end{Soutput}
\end{Schunk}






\newpage

\section{Adult Density: At the ninetieth percentile}
\begin{Schunk}
\begin{Soutput}
Call:
lm(formula = SimAbsDen ~ ExpAbsDen, data = PlotMeans)

Residuals:
    Min      1Q  Median      3Q     Max 
-760.57  -69.53  -46.39   -1.27  624.00 

Coefficients:
            Estimate Std. Error t value Pr(>|t|)    
(Intercept) 49.30528   21.00761   2.347   0.0209 *  
ExpAbsDen    0.34223    0.02421  14.133   <2e-16 ***
---
Signif. codes:  0 ‘***’ 0.001 ‘**’ 0.01 ‘*’ 0.05 ‘.’ 0.1 ‘ ’ 1

Residual standard error: 183.1 on 100 degrees of freedom
Multiple R-squared:  0.6664,	Adjusted R-squared:  0.6631 
F-statistic: 199.8 on 1 and 100 DF,  p-value: < 2.2e-16
\end{Soutput}
\end{Schunk}
\includegraphics{091015-g-014}

Now, how are the individual species doing?

\includegraphics{091015-g-015}
\begin{Schunk}
\begin{Sinput}
>   sppSlopes
\end{Sinput}
\begin{Soutput}
   species       ba90     ba100       ba110       den90
1     ABCO  3.0579913 2.8679859   2.9938315   1.5118002
2     ABMA  0.9690096 0.8968508   0.9284805   2.2299263
3     CADE  1.0242647 0.9240230   0.9865699   2.3914575
4     PICO 12.5267866 0.9295847   0.9950835   1.5878141
5     PIJE  2.6504097 2.4811348   3.1278394  12.0501592
6     PILA  2.6446403 2.5907967   2.5950433   3.6348070
7     PIMO -0.1743235 0.9567542   0.9179473   4.2058472
8     PIPO  0.5949587 0.6919119   0.6106893   0.4875871
9     QUCH  0.5687714 7.0914178 -12.7792250 -15.9249908
10    QUKE  1.0520175 1.0172449   1.0209050   2.6351695
\end{Soutput}
\end{Schunk}


\newpage
\section{At the original parameter designation}
\begin{Schunk}
\begin{Soutput}
Call:
lm(formula = SimAbsDen ~ ExpAbsDen, data = PlotMeans)

Residuals:
    Min      1Q  Median      3Q     Max 
-776.11  -73.57  -48.99   -4.74  687.39 

Coefficients:
            Estimate Std. Error t value Pr(>|t|)    
(Intercept) 52.13734   22.09152    2.36   0.0202 *  
ExpAbsDen    0.34780    0.02546   13.66   <2e-16 ***
---
Signif. codes:  0 ‘***’ 0.001 ‘**’ 0.01 ‘*’ 0.05 ‘.’ 0.1 ‘ ’ 1

Residual standard error: 192.5 on 100 degrees of freedom
Multiple R-squared:  0.651,	Adjusted R-squared:  0.6475 
F-statistic: 186.6 on 1 and 100 DF,  p-value: < 2.2e-16
\end{Soutput}
\end{Schunk}
\includegraphics{091015-g-017}

\includegraphics{091015-g-018}
\begin{Schunk}
\begin{Sinput}
>   sppSlopes
\end{Sinput}
\begin{Soutput}
   species       ba90     ba100       ba110       den90     den100
1     ABCO  3.0579913 2.8679859   2.9938315   1.5118002   1.366993
2     ABMA  0.9690096 0.8968508   0.9284805   2.2299263   2.218563
3     CADE  1.0242647 0.9240230   0.9865699   2.3914575   2.345902
4     PICO 12.5267866 0.9295847   0.9950835   1.5878141   1.479854
5     PIJE  2.6504097 2.4811348   3.1278394  12.0501592  27.477350
6     PILA  2.6446403 2.5907967   2.5950433   3.6348070   3.557684
7     PIMO -0.1743235 0.9567542   0.9179473   4.2058472   4.056143
8     PIPO  0.5949587 0.6919119   0.6106893   0.4875871   1.296860
9     QUCH  0.5687714 7.0914178 -12.7792250 -15.9249908 -15.618990
10    QUKE  1.0520175 1.0172449   1.0209050   2.6351695   2.577005
\end{Soutput}
\end{Schunk}





\newpage
\section{At the one hundred and tenth percentile}
\begin{Schunk}
\begin{Soutput}
Call:
lm(formula = SimAbsDen ~ ExpAbsDen, data = PlotMeans)

Residuals:
    Min      1Q  Median      3Q     Max 
-768.33  -73.27  -50.11   -4.32  703.03 

Coefficients:
            Estimate Std. Error t value Pr(>|t|)    
(Intercept) 52.91528   22.24782   2.378   0.0193 *  
ExpAbsDen    0.35061    0.02564  13.672   <2e-16 ***
---
Signif. codes:  0 ‘***’ 0.001 ‘**’ 0.01 ‘*’ 0.05 ‘.’ 0.1 ‘ ’ 1

Residual standard error: 193.9 on 100 degrees of freedom
Multiple R-squared:  0.6515,	Adjusted R-squared:  0.648 
F-statistic: 186.9 on 1 and 100 DF,  p-value: < 2.2e-16
\end{Soutput}
\end{Schunk}
\includegraphics{091015-g-020}

\includegraphics{091015-g-021}
\begin{Schunk}
\begin{Sinput}
>   sppSlopes
\end{Sinput}
\begin{Soutput}
   species       ba90     ba100       ba110       den90     den100     den110
1     ABCO  3.0579913 2.8679859   2.9938315   1.5118002   1.366993   1.351460
2     ABMA  0.9690096 0.8968508   0.9284805   2.2299263   2.218563   2.222191
3     CADE  1.0242647 0.9240230   0.9865699   2.3914575   2.345902   2.317878
4     PICO 12.5267866 0.9295847   0.9950835   1.5878141   1.479854   1.570408
5     PIJE  2.6504097 2.4811348   3.1278394  12.0501592  27.477350   4.776453
6     PILA  2.6446403 2.5907967   2.5950433   3.6348070   3.557684   3.485326
7     PIMO -0.1743235 0.9567542   0.9179473   4.2058472   4.056143   3.964480
8     PIPO  0.5949587 0.6919119   0.6106893   0.4875871   1.296860   1.428173
9     QUCH  0.5687714 7.0914178 -12.7792250 -15.9249908 -15.618990 -21.340777
10    QUKE  1.0520175 1.0172449   1.0209050   2.6351695   2.577005   2.609140
\end{Soutput}
\begin{Sinput}
>   write.csv(sppSlopes, file=paste(parName, ".csv", sep=""))
\end{Sinput}
\end{Schunk}
\end{document}
